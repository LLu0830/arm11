% This is a template for students in MATH 3000 at FSU to formally write-up results presented in class.

% All of this stuff with '%' in front is a comment and ignored by the compiler.
%
% The lines before the "\begin{document}" line is called the preamble.
% This is where you load particular packages you need.
% Until you are more experienced, or the program says you are missing packages, it is safe to ignore most of the preamble.
%
%----------------------------------

\documentclass[12pt]{article}
\usepackage[margin=1in]{geometry}% Change the margins here if you wish.
\setlength{\parindent}{0pt} % This is the set the indent length for new paragraphs, change if you want.
\setlength{\parskip}{5pt} % This sets the distance between paragraphs, which will be used anytime you have a blank line in your LaTeX code.
\pagenumbering{gobble}% This means the page will not be numbered. You can comment it out if you like page numbers.

%------------------------------------

% These packages allow the most of the common "mathly things"
\usepackage{amsmath,amsthm,amssymb}

% This package allows you to add images.
\usepackage{graphicx}
\usepackage{float}

% These are theorem environments.  This should cover everything you need, and you should be able to tell what environment goes with what type of result, but please let me know if I've missed anything.
\newtheorem{euclidtheorem}{Proposition}[section]
\newtheorem{classtheorem}{Theorem}
\newtheorem{theorem}{Theorem}[section]
\newtheorem{challenge}[theorem]{Challenge}
\newtheorem{question}[theorem]{Question}
\newtheorem{problem}[theorem]{Problem}


\newtheorem{theorempiece}{Theorem}[theorem]
\newtheorem{classtheorempiece}{Theorem}[classtheorem]
\newtheorem{challengepiece}{Challenge}[theorem]
\newtheorem{questionpiece}{Question}[theorem]
\newtheorem{problempiece}{Problem}[theorem]

%These help to format the names of the results the way we are in class and in notes.
\renewcommand*{\theeuclidtheorem}{\Roman{section}.\arabic{theorem}}
\renewcommand*{\thetheorem}{\arabic{section}.\arabic{theorem}}
\renewcommand*{\theclasstheorem}{\Alph{theorem}}
\renewcommand*{\thechallenge}{\arabic{section}.\arabic{theorem}}
\renewcommand*{\thequestion}{\arabic{section}.\arabic{theorem}}
\renewcommand*{\theproblem}{\arabic{section}.\arabic{theorem}}
\renewcommand*{\thetheorempiece}{\arabic{section}.\arabic{theorem}.\alph{theorempiece}}
\renewcommand*{\thechallengepiece}{\arabic{section}.\arabic{theorem}.\alph{challengepiece}}
\renewcommand*{\thequestionpiece}{\arabic{section}.\arabic{theorem}.\alph{questionpiece}}
\renewcommand*{\theproblempiece}{\arabic{section}.\arabic{theorem}.\alph{problempiece}}


% Should you need any additional packages, you can load them here. If you've looked up something (like on DeTeXify), it should specify if you need a special package.  Just copy and paste what is below, and put the package name in the { }.
\usepackage{wasysym} %this lets me make smiley faces :-)

% Put the name of your paper here. It does not need to be a fancy name, but should tell the reader what is contained in the paper.
\title{C Project - Checkpoint Report}

% You are the author, put your name here.
\author{Rini Banerjee\\Katarina Kulkova\\Lize Lu\\Manshu Wang}

% You can change the date to be something other than the current date if you want.
\date{Group 31}

\begin{document}

\maketitle

%Write a (very brief) introduction here.  You should include the name of the original presenter of the result, and (if appropriate) the name of the person who posed the question or conjecture originally.  Also include the names of any other helpful persons to your write-up, the presentation, or the solution itself.

%Include some narrative that will connect this result to others or give some of the background on the questions or conjectures the result answers.  (This will be easier in some cases than others; don't worry too much about being eloquent here \dots focus on the art of proof-writing first.)

Conjecture A was Ben's rephrasing of a portion of the original Conjecture 1.1.  Andrew and Jake both gave initial ideas for the proof on 1/23 and Elijah presented a formal proof on 1/25.

The other half of Conjecture 1.1 was determined to be untrue and possibly corrected by Gianna into Conjecture B which is still to be proven at the time of this writing.

% These set the counters for the section (of our notes; so section 1 is rhombuses, section 2 is kites, etc.) and theorem (or problem, challenge, etc.)  If you are only presenting a piece of a theorem that is not the first piece, you'll need to un-comment the last counter and adjust it appropriately.
\setcounter{section}{1}
\setcounter{theorem}{1}
%\setcounter{theorempiece}{2}


% Use the appropriate environment to get the name that you want.
% I'm using classtheorem so that the appropriate letter will appear.
\begin{classtheorem}
The opposite angles in a rhombus are congruent.
\end{classtheorem}
% Every environment should have a \begin{} and \end{} where the inside of {} matches.
\begin{proof} % This environment prints "Proof." at the beginning and the box at the end.
Let $ABCD$ be a rhombus.  Our aim is to show that each pair of opposite angles are congruent.  Without loss of generality, we choose to show that angle $ABC$ is congruent to angle $CDA$.

% Notice that stuff inside $ $ looks different in the pdf.  This is math mode, and you should use it for the names of mathematical objects, and you HAVE to use it for almost every mathematical symbol.

Use Euclid's Postulate 2, construct the diagonal $AC$.  This creates two triangles and we aim to show these triangles are congruent.

First notice that segments $AB$ and $CD$ are congruent by the definition of a rhombus.  Similarly, segments $BC$ and $DA$ are congruent to one another.  Lastly the diagonal segment $AC$ is congruent to segment $AC$ by Euclid's Common Notion 4.

\setcounter{figure}{0}
\begin{figure}[H]% This environment labels and generates a figure separate from the text.  I used the [H] to place the figure exactly where I wanted it.  If this isn't important to the reading and understanding of your proof, you can let LaTeX choose where to put it by leaving out the [] altogether.  You can use other tags to control the placement of a figure, [t] for the top of a page and [b] for the bottom are convenient sometimes.
\centering
\includegraphics[width = .4\textwidth]{ThmA_picture}
% This is how we include pictures in a LaTeX document.  In Overleaf, you need to add this file to the project.  If this window with code is all the way on the left, click on the tiny > on the left.  Click on the upload icon near the top (with the upward arrow). Upload the picture file(s) from your computer. (If you have installed a LaTeX editor on your own machine, then make sure the file you want to include is in the same folder as your .tex document... or look up how to call other folders.)
% Once the file is part of the project, type the name of the file inside the {}.  You typically do not need the file type .pdf, .jpg, etc. but nothing bad happens when you include it.
% The [] can be left empty or used to enter a variety of different instructions, most commonly adjustments to the size of the image are here. You can use inches or centimeters or points, LaTeX knows many measurement systems. You can also specify relative lengths.  \textwidth is the width of the text in this document, and putting .4 in front shrinks it to 40% of the text width.  You can also give a scale rather than a width or height: [scale = .5] will shrink the entire image to half the size of the original.
\label{fig:A.1}
% This is a label we can use to reference this figure... more helpful if you have lots of figures in your proof.
\caption{$\overline{AB} \cong \overline{CD}, \overline{BC} \cong \overline{DA}, \overline{AC} \cong \overline{AC}$}
% I like to caption the pictures in my geometry proofs with the congruences that are "constructed in" to the figure.  I basically include the "givens" and allow the reader to verify the conclusions by reading the proof and looking at the diagram together.  You can of course make different choices.
\end{figure}

All together, that gives three pairs of corresponding sides of each triangle congruent to one another.  By Euclid I.8, triangles $ABC$ and $CDA$ are congruent, and the corresponding angles, namely angles $ABC$ and $CDA$ are congruent.

Since our choice of a pair of opposite angles was arbitrary, we have proven our claim.

\end{proof}

If there is any narrative that makes sense here, or closing remarks, feel free to include something.  $\smiley$



\end{document}
